\documentclass{article}

\usepackage{booktabs}
\usepackage{tabularx}

\title{Development Plan\\\progname}

\author{\authname}

\date{}

%% Comments

\usepackage{color}

\newif\ifcomments\commentstrue %displays comments
%\newif\ifcomments\commentsfalse %so that comments do not display

\ifcomments
\newcommand{\authornote}[3]{\textcolor{#1}{[#3 ---#2]}}
\newcommand{\todo}[1]{\textcolor{red}{[TODO: #1]}}
\else
\newcommand{\authornote}[3]{}
\newcommand{\todo}[1]{}
\fi

\newcommand{\wss}[1]{\authornote{blue}{SS}{#1}} 
\newcommand{\plt}[1]{\authornote{magenta}{TPLT}{#1}} %For explanation of the template
\newcommand{\an}[1]{\authornote{cyan}{Author}{#1}}

%% Common Parts

\newcommand{\progname}{ProgName} % PUT YOUR PROGRAM NAME HERE
\newcommand{\authname}{Team \#, Team Name
\\ Student 1 name
\\ Student 2 name
\\ Student 3 name
\\ Student 4 name} % AUTHOR NAMES                  

\usepackage{hyperref}
    \hypersetup{colorlinks=true, linkcolor=blue, citecolor=blue, filecolor=blue,
                urlcolor=blue, unicode=false}
    \urlstyle{same}
                                


\begin{document}

\begin{table}[hp]
\caption{Revision History} \label{TblRevisionHistory}
\begin{tabularx}{\textwidth}{llX}
\toprule
\textbf{Date} & \textbf{Developer(s)} & \textbf{Change}\\
\midrule
Date1 & Name(s) & Description of changes\\
Date2 & Name(s) & Description of changes\\
... & ... & ...\\
\bottomrule
\end{tabularx}
\end{table}

\newpage

\maketitle

Our team: Controls Freaks, has had a long love for 
the world of control systems. Our mission is to develop 
a control system for McMaster’s Formula FSAE team, to learn 
and better understand how vehicle’s and similar systems operate, 
and to cohesively work in a team to bring together the knowledge 
we have gained over our university careers into one final project. 
Our team is primarily mechatronics based, with four members 
(Laura, Jason, Abhishek, and Derek) being in the program, while 
our final member, Dharak, brings expertise from the computer engineering 
program and an understanding of the McMaster team's vehicle sub-systems.

\section{Team Meeting Plan}
Team meetings will be held once a week on 
Thursdays at 8:00 pm and will go for roughly 
one hour. Ad hoc meetings will be held as needed.

\section{Team Communication Plan}

A discord server has been created for team communication. This will be 
the primary form of communication for the team meetings as the application 
has strong functionalities for desktop video calls. For more general messages, 
a Facebook messenger group chat has been created to update each other on any 
ideas/issues that arise day to day.

\section{Team Member Roles}

Derek - Motor Controller & Vehicle Dynamics Specialist
Laura - Driver Inputs Specialist
Abhishek - Simulation & Validation Specialist
Jason - Battery Management System Specialist
Dharak - Team Lead and Inertial Sensors Specialist

\section{Workflow Plan}

The public Github repository will be the location where the project is.
Since there are multiple subsystems that will be worked on synchronously,
new branches will be created. Our convention used for branch naming will be
“Subsystem Name - Major Feature Being Added”. An example for this will be 
“Battery Management System - CAN Signals”. Once a significant feature has 
reached sufficient progress, a pull request will be made and a different 
team member will review the change. If there are contentions made by the 
reviewer, a comment will be added to the pull request. If it has no issues 
it will be merged into the main branch.


To make it easier to track updates to the branches, detailed comments will 
also be required from each team member on all commits. The frequency of 
commits will not be strictly monitored but the general rule of thumb is to
commit every 30 min (assuming continuous lines of code were being typed), 
or once a subfeature has been implemented (e.g a bug regarding device 
connectivity has been fixed). 

Subfeature tasks will be tracked and managed on Kanban board (Github 
project). Every week there will be a meeting to decide on the tasks for the
upcoming week (and potentially beyond), Tasks will then be assigned to the 
appropriate team member. Any issues or questions regarding the tasks can be 
added as a comment to the specific task on the board.

\begin{itemize}
	\item How will you be using git, including branches, pull request, etc.?
	\item How will you be managing issues, including template issues, issue
	classificaiton, etc.?
\end{itemize}

\section{Proof of Concept Demonstration Plan}

What is the main risk, or risks, for the success of your project?  What will you
demonstrate during your proof of concept demonstration to convince yourself that
you will be able to overcome this risk?

\section{Technology}

\begin{itemize}
\item Specific programming language
\item Specific linter tool (if appropriate)
\item Specific unit testing framework
\item Investigation of code coverage measuring tools
\item Specific plans for Continuous Integration (CI), or an explanation that CI
  is not being done
\item Specific performance measuring tools (like Valgrind), if
  appropriate
\item Libraries you will likely be using?
\item Tools you will likely be using?
\end{itemize}

\section{Coding Standard}

\section{Project Scheduling}

\wss{How will the project be scheduled?}

\end{document}