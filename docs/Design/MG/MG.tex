\documentclass[12pt, titlepage]{article}

\usepackage{fullpage}
\usepackage[round]{natbib}
\usepackage{multirow}
\usepackage{booktabs}
\usepackage{tabularx}
\usepackage{graphicx}
\usepackage{float}
\usepackage{hyperref}
\hypersetup{
    colorlinks,
    citecolor=blue,
    filecolor=black,
    linkcolor=red,
    urlcolor=blue
}

\input{../../Comments}
%% Common Parts

\newcommand{\progname}{Mechatronics} % PUT YOUR PROGRAM NAME HERE
\newcommand{\authname}{Team 28, Controls Freaks
\\ Abhishek Magdum
\\ Dharak Verma
\\ Jason Surendran
\\ Laura Yang
\\ Derek Paylor} % AUTHOR NAMES                  

\usepackage{hyperref}
    \hypersetup{colorlinks=true, linkcolor=blue, citecolor=blue, filecolor=blue,
                urlcolor=blue, unicode=false}
    \urlstyle{same}
                                

\newcounter{acnum}
\newcommand{\actheacnum}{AC\theacnum}
\newcommand{\acref}[1]{AC\ref{#1}}

\newcounter{ucnum}
\newcommand{\uctheucnum}{UC\theucnum}
\newcommand{\uref}[1]{UC\ref{#1}}

\newcounter{mnum}
\newcommand{\mthemnum}{M\themnum}
\newcommand{\mref}[1]{M\ref{#1}}

\begin{document}

\title{Module Guide for \progname{}} 
\author{\authname}
\date{\today}

\maketitle

\pagenumbering{roman}

\section{Revision History}

\begin{tabularx}{\textwidth}{p{3cm}p{2cm}X}
\toprule {\bf Date} & {\bf Version} & {\bf Notes}\\
\midrule
January 18, 2023 & 1.0 & First Version\\
\bottomrule
\end{tabularx}

\newpage

\section{Reference Material}

This section records information for easy reference.

\subsection{Abbreviations and Acronyms}

\renewcommand{\arraystretch}{1.2}
\begin{tabular}{l l} 
  \toprule		
  \textbf{symbol} & \textbf{description}\\
  \midrule 
  AC & Anticipated Change\\
  DAG & Directed Acyclic Graph \\
  M & Module \\
  MG & Module Guide \\
  OS & Operating System \\
  R & Requirement\\
  SC & Scientific Computing \\
  SRS & Software Requirements Specification\\
  \progname & Explanation of program name\\
  UC & Unlikely Change \\
  \bottomrule
\end{tabular}\\

\newpage

\tableofcontents

\listoftables

\listoffigures

\newpage

\pagenumbering{arabic}

\section{Introduction}

Decomposing a system into modules is a commonly accepted approach to developing
software.  A module is a work assignment for a programmer or programming
team~\citep{ParnasEtAl1984}.  We advocate a decomposition
based on the principle of information hiding~\citep{Parnas1972a}.  This
principle supports design for change, because the ``secrets'' that each module
hides represent likely future changes.  Design for change is valuable in SC,
where modifications are frequent, especially during initial development as the
solution space is explored.  

Our design follows the rules layed out by \citet{ParnasEtAl1984}, as follows:
\begin{itemize}
\item System details that are likely to change independently should be the
  secrets of separate modules.
\item Each data structure is implemented in only one module.
\item Any other program that requires information stored in a module's data
  structures must obtain it by calling access programs belonging to that module.
\end{itemize}

After completing the first stage of the design, the Software Requirements
Specification (SRS), the Module Guide (MG) is developed~\citep{ParnasEtAl1984}. The MG
specifies the modular structure of the system and is intended to allow both
designers and maintainers to easily identify the parts of the software.  The
potential readers of this document are as follows:

\begin{itemize}
\item New project members: This document can be a guide for a new project member
  to easily understand the overall structure and quickly find the
  relevant modules they are searching for.
\item Maintainers: The hierarchical structure of the module guide improves the
  maintainers' understanding when they need to make changes to the system. It is
  important for a maintainer to update the relevant sections of the document
  after changes have been made.
\item Designers: Once the module guide has been written, it can be used to
  check for consistency, feasibility, and flexibility. Designers can verify the
  system in various ways, such as consistency among modules, feasibility of the
  decomposition, and flexibility of the design.
\end{itemize}

The rest of the document is organized as follows. Section
\ref{SecChange} lists the anticipated and unlikely changes of the software
requirements. Section \ref{SecMH} summarizes the module decomposition that
was constructed according to the likely changes. Section \ref{SecConnection}
specifies the connections between the software requirements and the
modules. Section \ref{SecMD} gives a detailed description of the
modules. Section \ref{SecTM} includes two traceability matrices. One checks
the completeness of the design against the requirements provided in the SRS. The
other shows the relation between anticipated changes and the modules. Section
\ref{SecUse} describes the use relation between modules.

\section{Anticipated and Unlikely Changes} \label{SecChange}

This section lists possible changes to the system. According to the likeliness
of the change, the possible changes are classified into two
categories. Anticipated changes are listed in Section \ref{SecAchange}, and
unlikely changes are listed in Section \ref{SecUchange}.

\subsection{Anticipated Changes} \label{SecAchange}

Anticipated changes are the source of the information that is to be hidden
inside the modules. Ideally, changing one of the anticipated changes will only
require changing the one module that hides the associated decision. The approach
adopted here is called design for change. The only anticipated change would be changes to the 
torque vectoring and traction control logic, that lives inside of the vehicle dynamics block.

%gotta fix 
\begin{description}
\item[\refstepcounter{acnum} \actheacnum \label{acTV}:] Torque Vectoring
\item[\refstepcounter{acnum} \actheacnum \label{acTC}:] Traction Control
\end{description}

%assuming every input is correct

\subsection{Unlikely Changes} \label{SecUchange}

The module design should be as general as possible. However, a general system is
more complex. Sometimes this complexity is not necessary. Fixing some design
decisions at the system architecture stage can simplify the software design. If
these decision should later need to be changed, then many parts of the design
will potentially need to be modified. Hence, it is not intended that these
decisions will be changed.

\begin{description}
\item[\refstepcounter{ucnum} \uctheucnum \label{ucIO}:] Input/Output devices
  (Input: Pedals, steering sensors, motor and contactor feedback, Output: Speaker, motor commands).
\item[\refstepcounter{ucnum} \uctheucnum \label{ucFSAE}:] FSAE rules and regulations
\item[\refstepcounter{ucnum} \uctheucnum \label{ucModules}:] Module heirarchy
\end{description}

\section{Module Hierarchy} \label{SecMH}

This section provides an overview of the module design. Modules are summarized
in a hierarchy decomposed by secrets in Table \ref{TblMH}. The modules listed
below, which are leaves in the hierarchy tree, are the modules that will
actually be implemented.

\begin{description}
\item [\refstepcounter{mnum} \mthemnum \label{mDI}:] Driver Interface
\item [\refstepcounter{mnum} \mthemnum \label{mVD}:] Vehicle Dynamics
\item [\refstepcounter{mnum} \mthemnum \label{mMI}:] Motor Interface
\item [\refstepcounter{mnum} \mthemnum \label{mBM}:] Battery Monitor
\item [\refstepcounter{mnum} \mthemnum \label{mG}:] Governor
\item [\refstepcounter{mnum} \mthemnum \label{mPlant}:] Plant Model
\end{description}


\begin{table}[h!]
\centering
\begin{tabular}{p{0.3\textwidth} p{0.6\textwidth}}
\toprule
\textbf{Level 1} & \textbf{Level 2}\\
\midrule

{Hardware-Hiding Module} & Plant Model \\
\midrule

\multirow{7}{0.3\textwidth}{Behaviour-Hiding Module} & Driver Interface\\
& Vehicle Dynamics\\
& Motor Interface\\
& Battery Monitor\\
\midrule

\multirow{3}{0.3\textwidth}{Software Decision Module} & \\
& Governor\\
\bottomrule

\end{tabular}
\caption{Module Hierarchy}
\label{TblMH}
\end{table}

\section{Connection Between Requirements and Design} \label{SecConnection}

The design of the system is intended to satisfy the requirements developed in
the SRS. In this stage, the system is decomposed into modules. The connection
between requirements and modules is listed in Table~\ref{TblRT}.

\section{Module Decomposition} \label{SecMD}

Modules are decomposed according to the principle of ``information hiding''
proposed by \citet{ParnasEtAl1984}. The \emph{Secrets} field in a module
decomposition is a brief statement of the design decision hidden by the
module. The \emph{Services} field specifies \emph{what} the module will do
without documenting \emph{how} to do it. For each module, a suggestion for the
implementing software is given under the \emph{Implemented By} title. If the
entry is \emph{OS}, this means that the module is provided by the operating
system or by standard programming language libraries.  \emph{\progname{}} means the
module will be implemented by the \progname{} software.

Only the leaf modules in the hierarchy have to be implemented. If a dash
(\emph{--}) is shown, this means that the module is not a leaf and will not have
to be implemented.

\subsection{Hardware Hiding Modules}

\begin{description}
\item[Secrets:]The data structure and algorithm used to implement the virtual
  hardware.
\item[Services:]Serves as a virtual hardware used by the rest of the
  system. This module provides the interface between the hardware and the
  software. So, the system can use it to display outputs or to accept inputs.
\item[Implemented By:] OS
\end{description}

\subsubsection{Plant Module (\mref{mPlant})}
\begin{description}
\item[Secrets:]Drive input signals, HV and LV contactors.
\item[Services:]This module simulates the physical input signals to the various other models in order to test various functionality and edge cases.
\item[Implemented By:] driver\_plant.slx
\item[Type of Module:] Simulink Model
\end{description}

\subsection{Behaviour-Hiding Module}

\subsubsection{Driver Interface Module (\mref{mDI})}
\begin{description}
\item[Secrets:]
\item[Services:]Pedal mapping
\item[Implemented By:] driver\_interface\_lib.slx
\item[Type of Module:] Simulink Model
\end{description}

\subsubsection{Vehicle Dynamics Module (\mref{mVD})}
\begin{description}
\item[Secrets:]The format and structure of the input data. ****check this
\item[Services:]Converts the input data into the data structure used by the
  input parameters module.
\item[Implemented By:] vehicle\_dynamics\_lib.slx
\item[Type of Module:] Simulink Model
\end{description}

\subsubsection{Motor Interface Module (\mref{mMI})}
\begin{description}
\item[Secrets:]
\item[Services:]Controls the left and right motors
  input parameters module.
\item[Implemented By:] motor\_interface\_lib.slx
\item[Type of Module:] Simulink Model
\end{description}

\subsubsection{Battery Monitor Module (\mref{mBM})}
\begin{description}
\item[Secrets:]The format and structure of the input data. ****check this
\item[Services:]Takes in contactor statuses for the battery and outputs battery status to the governor.
\item[Implemented By:] battery\_monitor\_lib.slx
\item[Type of Module:] Simulink Model
\end{description}

\subsubsection{Governor Module (\mref{mG})}
\begin{description}
\item[Secrets:]The format and structure of the input data. ****check this
\item[Services:]Monitors inputs from Driver Interface, Vehicle Dynamics, Motor Interface and Battery Monitor modules and sends commands to these modules based on the current state of the vehicle.
\item[Implemented By:] governor\_lib.slx
\item[Type of Module:] Simulink Model
\end{description}

\section{Traceability Matrix} \label{SecTM}

This section shows two traceability matrices: between the modules and the
requirements and between the modules and the anticipated changes.

% the table should use mref, the requirements should be named, use something
% like fref
\begin{table}[H]
\centering
\begin{tabular}{p{0.6\textwidth} p{0.2\textwidth}}
\toprule
\textbf{Req.} & \textbf{Modules}\\
\midrule
NFR1, NFR2, NFR3, NFR4, NFR5, NFR6 & \mref{mDI}\\
\midrule
VDR1, VDR2, NFR1, NFR2, NFR3, NFR5, NFR6 & \mref{mVD}\\
\midrule
TMR1, TMR2, TMR2, TMR4, TMR5, TMR6, TMR7, NFR1, NFR2, NFR3, NFR4, NFR5, NFR6 & \mref{mMI}\\
\midrule
AMR1, AMR2, AMR3, AMR4, AMR5, AMR6, AMR7, AMR8, AMR9, NFR1, NFR2, NFR3, NFR5, NFR6 & \mref{mBM} \\
\midrule
MSR1, MSR2, MSR3, MSR4, MSR5, MSR6, MSR7, MSR8, MSR9, NFR1, NFR2, NFR3, NFR5, NFR6 & \mref{mG} \\
\bottomrule
\end{tabular}
\caption{Trace Between Requirements and Modules}
\label{TblRT}
\end{table}

\section{Use Hierarchy Between Modules} \label{SecUse}

In this section, the uses hierarchy between modules is
provided. \citet{Parnas1978} said of two programs A and B that A {\em uses} B if
correct execution of B may be necessary for A to complete the task described in
its specification. That is, A {\em uses} B if there exist situations in which
the correct functioning of A depends upon the availability of a correct
implementation of B.  Figure \ref{FigUH} illustrates the use relation between
the modules. It can be seen that the graph is a directed acyclic graph
(DAG). Each level of the hierarchy offers a testable and usable subset of the
system, and modules in the higher level of the hierarchy are essentially simpler
because they use modules from the lower levels.

\begin{figure}[H]
\centering
\includegraphics[width=\textwidth]{heirarchy.png}
\caption{Use hierarchy among modules}
\label{FigUH}
\end{figure}

%\section*{References}

\bibliographystyle {plainnat}
\bibliography{../../../refs/References}

\newpage{}

\end{document}