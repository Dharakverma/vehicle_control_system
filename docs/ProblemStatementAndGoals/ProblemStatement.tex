\documentclass{article}

\usepackage{tabularx}
\usepackage{booktabs}

\title{Problem Statement and Goals\\\progname}

\author{\authname}

\date{}

%% Comments

\usepackage{color}

\newif\ifcomments\commentstrue %displays comments
%\newif\ifcomments\commentsfalse %so that comments do not display

\ifcomments
\newcommand{\authornote}[3]{\textcolor{#1}{[#3 ---#2]}}
\newcommand{\todo}[1]{\textcolor{red}{[TODO: #1]}}
\else
\newcommand{\authornote}[3]{}
\newcommand{\todo}[1]{}
\fi

\newcommand{\wss}[1]{\authornote{blue}{SS}{#1}} 
\newcommand{\plt}[1]{\authornote{magenta}{TPLT}{#1}} %For explanation of the template
\newcommand{\an}[1]{\authornote{cyan}{Author}{#1}}

%% Common Parts

\newcommand{\progname}{ProgName} % PUT YOUR PROGRAM NAME HERE
\newcommand{\authname}{Team \#, Team Name
\\ Student 1 name
\\ Student 2 name
\\ Student 3 name
\\ Student 4 name} % AUTHOR NAMES                  

\usepackage{hyperref}
    \hypersetup{colorlinks=true, linkcolor=blue, citecolor=blue, filecolor=blue,
                urlcolor=blue, unicode=false}
    \urlstyle{same}
                                


\begin{document}

\maketitle

\begin{table}[hp]
\caption{Revision History} \label{TblRevisionHistory}
\begin{tabularx}{\textwidth}{llX}
\toprule
\textbf{Date} & \textbf{Developer(s)} & \textbf{Change}\\
\midrule
Date1 & Name(s) & Description of changes\\
Date2 & Name(s) & Description of changes\\
... & ... & ...\\
\bottomrule
\end{tabularx}
\end{table}

\section{Overview}
Machines are designed by humans, for humans, 
to make everyday life easier. Until a machine has a 
method to control it, either through human control or 
autonomous control, they are nothing more than a paperweight. 
In our everyday world, cars are a ubiquitous machine used to 
transport people and goods to their destination, and thereby 
are a key pillar of a productive economy and society. In the 
21st century, vehicles are becoming more interconnected and 
sophisticated than ever and use computers to control almost 
every subsystem of the vehicle. Control systems encapsulate 
the “brains” behind such machines, allowing them to interpret 
their environment using sensors, determine a desired state, and 
manipulate the environment using actuators to achieve the 
desired state. They convert a physical stimulus (from the 
environment, including a user) into a control signal for a 
component. Some examples of control systems seen in modern 
cars today are engine/transmission controls, HVAC controls,
battery management (in hybrid/electric vehicles, anti-lock 
braking (ABS) and electronic stability program (ESP).

\subsection{Project Description}
We are aiming to design, simulate, implement, and test a vehicle 
control system for a quarter-scale Formula 1 style electric vehicle. 
The control system will allow the vehicle to be operated in basic driving 
conditions by managing the following vehicle subsystems: battery management, 
cooling, vehicle mode selection, tractive motor, and vehicle dynamics. 
This capstone will be in collaboration with the MAC Formula Electric FSAE team, 
where we will be responsible for providing them with a suitable control 
system for their vehicle within competition rule specifications. We will 
be working with the embedded hardware provided by MAC Formula Electric for 
implementing and testing our control system once software-based simulation and 
validation is complete.

\subsection{Inputs and Outputs}

\wss{Characterize the problem in terms of ``high level'' inputs and outputs.  
Use abstraction so that you can avoid details.}

\subsection{Stakeholders}

\subsection{Environment}

\wss{Hardware and software}

\section{Goals}

\section{Stretch Goals}

\end{document}