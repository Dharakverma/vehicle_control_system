\documentclass{article}

\usepackage{tabularx}
\usepackage{booktabs}

\title{Problem Statement and Goals\\\progname}

\author{\authname}

\date{}

%% Comments

\usepackage{color}

\newif\ifcomments\commentstrue %displays comments
%\newif\ifcomments\commentsfalse %so that comments do not display

\ifcomments
\newcommand{\authornote}[3]{\textcolor{#1}{[#3 ---#2]}}
\newcommand{\todo}[1]{\textcolor{red}{[TODO: #1]}}
\else
\newcommand{\authornote}[3]{}
\newcommand{\todo}[1]{}
\fi

\newcommand{\wss}[1]{\authornote{blue}{SS}{#1}} 
\newcommand{\plt}[1]{\authornote{magenta}{TPLT}{#1}} %For explanation of the template
\newcommand{\an}[1]{\authornote{cyan}{Author}{#1}}

%% Common Parts

\newcommand{\progname}{ProgName} % PUT YOUR PROGRAM NAME HERE
\newcommand{\authname}{Team \#, Team Name
\\ Student 1 name
\\ Student 2 name
\\ Student 3 name
\\ Student 4 name} % AUTHOR NAMES                  

\usepackage{hyperref}
    \hypersetup{colorlinks=true, linkcolor=blue, citecolor=blue, filecolor=blue,
                urlcolor=blue, unicode=false}
    \urlstyle{same}
                                


\begin{document}

\maketitle

\begin{table}[hp]
\caption{Revision History} \label{TblRevisionHistory}
\begin{tabularx}{\textwidth}{llX}
\toprule
\textbf{Date} & \textbf{Developer(s)} & \textbf{Change}\\
\midrule
Date1 & Name(s) & Description of changes\\
Date2 & Name(s) & Description of changes\\
... & ... & ...\\
\bottomrule
\end{tabularx}
\end{table}

\section{Overview}
Machines are designed by humans, for humans, 
to make everyday life easier. Until a machine has a 
method to control it, either through human control or 
autonomous control, they are nothing more than a paperweight. 
In our everyday world, cars are a ubiquitous machine used to 
transport people and goods to their destination, and thereby 
are a key pillar of a productive economy and society. In the 
21st century, vehicles are becoming more interconnected and 
sophisticated than ever and use computers to control almost 
every subsystem of the vehicle. Control systems encapsulate 
the “brains” behind such machines, allowing them to interpret 
their environment using sensors, determine a desired state, and 
manipulate the environment using actuators to achieve the 
desired state. They convert a physical stimulus (from the 
environment, including a user) into a control signal for a 
component. Some examples of control systems seen in modern 
cars today are engine/transmission controls, HVAC controls, 
battery management (in hybrid/electric vehicles, anti-lock 
braking (ABS) and electronic stability program (ESP).

\subsection{Project Description}
We are aiming to design, simulate, implement, and test a vehicle 
control system for a quarter-scale Formula 1 style electric vehicle. 
The control system will allow the vehicle to be operated in basic driving 
conditions by managing the following vehicle subsystems: battery management, 
cooling, vehicle mode selection, tractive motor, and vehicle dynamics. 
This capstone will be in collaboration with the MAC Formula Electric FSAE team, 
where we will be responsible for providing them with a suitable control 
system for their vehicle within competition rule specifications. We will 
be working with the embedded hardware provided by MAC Formula Electric for 
implementing and testing our control system once software-based simulation and 
validation is complete.

\subsection{Inputs}
Inputs to the control system are sourced from the various hardware sub-systems 
of the vehicle, such as the battery management system, motor controller kit, 
inertial sensors (accelerometer, gyroscope), and driver inputs via steering 
and pedal sensors.Inputs from the battery management system regarding its 
current state, battery module temperatures, voltage output, and current draw 
must all be within safe operating specifications and thus will play a large 
role in the overall state of the vehicle control system. 
To comply with FSAE rulesets regarding the enabling of the tractive system, 
entering ready-to-drive mode, and safely shutting the vehicle down, the control 
system must also take input from the vehicle’s driver via physical push buttons 
mounted on the steering wheel. Lastly, the control system will be responsible 
for communication with the motor controllers which will be based on inputs from 
the driver via the pedal tray.All the aforementioned inputs to the control system 
will play a role in dictating the current state of the vehicle and its various 
sub-systems.

\subsection{Outputs}
Based on the hardware and system state inputs described above, our control system 
should output digital control signals for the vehicle’s actuators within each of 
the vehicle’s subsystems, such as battery, cooling, and motor controllers.Some 
vehicle systems work independently of each other. For example, the battery cooling 
loop must only enable once a set temperature is met, and fan speed is modulated 
based on heat generation and battery module temperatures. On the other hand, there 
are systems in the vehicle, for example the systems that control vehicle propulsion, 
that must work together in order to propel the vehicle. A simple push of the 
accelerator pedal must translate to various messages to the battery management system 
and motor controllers. These systems must be controlled and work in unisense to 
propel the vehicle forward. All of these considerations and limitations must be 
accounted for when programming of the vehicle control system commences.

\subsection{Stakeholders}

\subsection{Environment}

\wss{Hardware and software}

\section{Goals}

\begin{table}[hp]
\caption{Goals} \label{TblGoals}
\begin{tabularx}{\linewidth}{XX}
\toprule
\textbf{Goal} & \textbf{Description}\\
\midrule
Reliability of output & Ensure that state of the vehicle 
control system corresponds to the physical state of the 
vehicle including motor RPM, battery state of charge, and 
all driver inputs such as ready-to-drive button, emergency shutoff, 
and pedal inputs. Validate that the control system accurately 
takes feedback from the various vehicle sub-systems and commands 
state changes as required. Verify that all commanded sub-system 
state changes are as expected and can be replicated. \\
FSAE rules compliance & Ensure that the vehicle startup and 
shutdown sequence follow the FSAE rules. This will be achieved by 
creating flowcharts depicting the required state of each model based 
on user input and the rules-defined timing based state changes. We can 
then validate the output state changes of our control system with our 
flowchart for a 1:1 match.\\
Maintainability & We will ensure future maintainability of our control 
system by separating all sub-systems into individual simulink models and 
controlling them via an overarching state machine. For example, we can 
create a single model for controlling a motor, and then utilize that 
model multiple times to enable a four-wheel drive system. We will also 
ensure all our models are thoroughly documented both inside the Simulink 
model itself and via project documentation.\\
\bottomrule
\end{tabularx}
\end{table}

\section{Stretch Goals}

\begin{table}[hp]
\caption{Stretch Goals} \label{TblStretchGoals}
\begin{tabularx}{\textwidth}{XX}
\toprule
\textbf{Goal} & \textbf{Description}\\
\midrule
Vehicle Dynamics - Stability Control Program & Stability control is a great 
feature for keeping a 
vehicle on the road, and heading towards an intended direction, commanded 
by the driver. This is done by checking the position of the steering wheel, 
and comparing it to the z-axis rotation of the vehicle. When the driver 
commands a turn and the vehicle is slow to react, wheel slipping is detected. 
To reduce/eliminate the loss of traction, the vehicle can brake the wheels 
closest to the apex of a corner, to help apply a rotational moment on the 
vehicle and assist the driver to turn the vehicle around a corner. \\
Vehicle dynamics - torque vectoring & Torque vectoring works similar to a 
vehicle stability control program, except in this case, the creation of a 
moment on the z-axis of a vehicle is purely done through acceleration. This 
means that when acceleration of the vehicle is requested through the accelerator 
pedal, based on steering direction, the vehicle can vary torque to different sides 
of a vehicle. Usually when going around a corner, the torque will be applied to 
the wheels furthest from the corner apex, helping apply a moment on the z-axis of 
the vehicle to help achieve higher cornering velocities along with increased driver 
confidence.\\
Over the air data acquisition & Over-the-air data acquisition will allow us to 
gain vital vehicle data, while the vehicle is out running laps. This can help 
us diagnose any issues live, if we see any anomalies in the control system, 
increasing vehicle performance and safety.\\
\bottomrule
\end{tabularx}
\end{table}

\end{document}