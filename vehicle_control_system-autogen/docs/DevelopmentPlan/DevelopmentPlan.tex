\documentclass{article}

\usepackage{booktabs}
\usepackage{tabularx}

\title{Development Plan\\\progname}

\author{\authname}

\date{}

%% Comments

\usepackage{color}

\newif\ifcomments\commentstrue %displays comments
%\newif\ifcomments\commentsfalse %so that comments do not display

\ifcomments
\newcommand{\authornote}[3]{\textcolor{#1}{[#3 ---#2]}}
\newcommand{\todo}[1]{\textcolor{red}{[TODO: #1]}}
\else
\newcommand{\authornote}[3]{}
\newcommand{\todo}[1]{}
\fi

\newcommand{\wss}[1]{\authornote{blue}{SS}{#1}} 
\newcommand{\plt}[1]{\authornote{magenta}{TPLT}{#1}} %For explanation of the template
\newcommand{\an}[1]{\authornote{cyan}{Author}{#1}}

%% Common Parts

\newcommand{\progname}{ProgName} % PUT YOUR PROGRAM NAME HERE
\newcommand{\authname}{Team \#, Team Name
\\ Student 1 name
\\ Student 2 name
\\ Student 3 name
\\ Student 4 name} % AUTHOR NAMES                  

\usepackage{hyperref}
    \hypersetup{colorlinks=true, linkcolor=blue, citecolor=blue, filecolor=blue,
                urlcolor=blue, unicode=false}
    \urlstyle{same}
                                


\begin{document}

\begin{table}[hp]
\caption{Revision History} \label{TblRevisionHistory}
\begin{tabularx}{\textwidth}{XXX}
\toprule
\textbf{Date} & \textbf{Developer(s)} & \textbf{Change}\\
\midrule
September 25th, 2022 & Abhishek Magdum, Dharak Verma, Jason Surendran, Laura Yang, Derek Paylor & Initial document population\\
\bottomrule
\end{tabularx}
\end{table}

\newpage

\maketitle

Our team: Controls Freaks, has had a long love for 
the world of control systems. Our mission is to develop 
a control system for McMaster’s Formula FSAE team, to learn 
and better understand how vehicle’s and similar systems operate, 
and to cohesively work in a team to bring together the knowledge 
we have gained over our university careers into one final project. 
Our team is primarily mechatronics based, with four members 
(Laura, Jason, Abhishek, and Derek) being in the program, while 
our final member, Dharak, brings expertise from the computer engineering 
program and an understanding of the McMaster team's vehicle sub-systems.

\section{Team Meeting Plan}
Team meetings will be held once a week on 
Thursdays at 8:00 pm and will go for roughly 
one hour. Ad hoc meetings will be held as needed.

\section{Team Communication Plan}

A discord server has been created for team communication. This will be 
the primary form of communication for the team meetings as the application 
has strong functionalities for desktop video calls. For more general messages, 
a Facebook messenger group chat has been created to update each other on any 
ideas/issues that arise day to day.
\\
\section{Team Member Roles}

Derek - Motor Controller \& Vehicle Dynamics Specialist
Laura - Driver Inputs Specialist
Abhishek - Simulation \& Validation Specialist
Jason - Battery Management System Specialist
Dharak - Team Lead and Inertial Sensors Specialist

\section{Workflow Plan}

The public Github repository will be the location where the project is.
Since there are multiple subsystems that will be worked on synchronously,
new branches will be created. Our convention used for branch naming will be
“Subsystem Name - Major Feature Being Added”. An example for this will be 
“Battery Management System - CAN Signals”. Once a significant feature has 
reached sufficient progress, a pull request will be made and a different 
team member will review the change. If there are contentions made by the 
reviewer, a comment will be added to the pull request. If it has no issues 
it will be merged into the main branch.

To make it easier to track updates to the branches, detailed comments will 
also be required from each team member on all commits. The frequency of 
commits will not be strictly monitored but the general rule of thumb is to
commit every 30 min (assuming continuous lines of code were being typed), 
or once a subfeature has been implemented (e.g a bug regarding device 
connectivity has been fixed). 

Subfeature tasks will be tracked and managed on Kanban board (Github 
project). Every week there will be a meeting to decide on the tasks for the
upcoming week (and potentially beyond), Tasks will then be assigned to the 
appropriate team member. Any issues or questions regarding the tasks can be 
added as a comment to the specific task on the board.

\section{Proof of Concept Demonstration Plan}

For proof of concept, we’ll be able to demonstrate our vehicle control 
system through simulating inputs to our overarching state machine, which in 
turn will control our smaller subsystems and output the current vehicle state. 
All of this can be visualised through Matlab, using simulation tools already 
present. There are tools to help test and validate our control system, both 
through virtual models, which will be used to test our code early and often, 
and physical models, which we will use at the end to show off our final 
control system.


\section{Technology}
Embedded System: C
Simulation: Simulink
Control Model: Simulink
Battery Management System Hardware: Orion BMS
Motor Kit: AMK Motor Kit
Microcontroller: STM32F767ZI
No specific plans for a CI tool

\section{Coding Standard}
Github will be the version control application we will be using. This will allow us 
to continuously iterate, track changes, and store code. 

For this project, the majority of the codebase will exist as simulink (.slx) files. 
Due to this, automating testing through Github Actions will be tough, since much of 
the code will be unrecognisable to Github. To combat this issue, we will be enforcing 
a PR approval process that will require a minimum of two members to approve. We will 
be following a 'Scouts Honor' code where each potential PR approver must pull the 
branch to their local machine and validate the relevant changes before approving the 
merge. The individual requesting the code review cannot approve their own request.

As always, proper software development practices must be implemented at all points 
in the software development cycle. This means that code must be properly commented, 
naming conventions should make sense, code is easily readable, and proper function 
use should be implemented to ensure unit testing can be completed easily.

\section{Project Scheduling}
Major milestones will align with those given in the course outline. Project-specific 
milestones, such as control system features, testing or integration, will be considered 
for each subsystem, and tasks will be assigned during weekly meetings.

\end{document}